\chapter{Introduction}
Initiating the chapter, Artificial Intelligence and its allied areas are witnessing the rapid growth and advancements for removing the barrier between humans and machines for a better world and future ahead. This deals with the biomedical domain of technology very specifically with the brain tumours and their detection with traditional ways and a brief light to modern future approaches with the use of technology. The introduction starts with an overview of the problem statement which is an early-stage detection of brain tumours. Traditional approaches of detecting and factors associated with it. Image processing is the heart of the detection through Deep Learning and Machine learning models. The detection is done via deep learning methods comprising particular algorithms. Convolutional Neural Network is the heart of image analysis using machine learning or deep learning models which offers ease of analysis with good efficiency. Followed in the next section shows the significance of Computer Vision that can be helpful for image analysis. 
\section{Overvew of the detection system}

{Computer vision and image processing are a few of the most significant domains in upcoming technical advancements of the world. The main objective of computer vision and Image processing is to enable machines and devices to view and capture the world in the way humans see it. It is the entire process involving object detection, classification and analysing the results. Image processing has majorly two parts involved in the entire process. Pre-processing is the first step in image processing which consists of operations such as image enhancement, resizing, adjusting images. Post Processing comes as the second part in the process which has special modifications as per the needs - highlighting the segmented areas, removing noise, applying texts to the area needed. Thus, any image dataset which consists of similar fields can be processed and modified for particular needs to tackle the problems.}\\

{Artificial Intelligence and its effects have seen an exponential rise in advancements of the world through automation and smart technologies. Artificial Intelligence alongside machine learning and deep learning techniques have been embedded into almost every aspect of life. With the advent of Smart tools and systems using AI-ML-DL, both developers and consumers have been benefited in terms of better business deals and revenue generation. It is highly significant to today's highly sophisticated technical era of the world to make better decisions using such advanced systems and implement - integrate them into their real-life as per their needs.}\\

{In this paper, a system for the early-stage detection of brain tumours is proposed. As the human brain is the most important body organ and its improper working or disease can lead to mortality death with tumours. Hence a system based on image segmentation and the deep learning algorithm is proposed for early-stage detection of these tumours. }\\

{Brain tumours or neoplasm are basically the growth or mass of abnormal cells which grow inside the Brain. There has been an unprecedented growth in the number of humans diagnosed with Brain Tumours. There are two prominent types of brain Tumours one is benign ( non - cancerous ) and another one is Malignant ( cancerous ). There are various methodologies to detect them and validate the same with tissue tests. Traditional approaches to detecting brain tumours and treating them cost a lot of money and time, which is in most cases not suitable for middle-class men and dependents. All these factors contribute as disadvantages and thus it is beneficial to bring technology into consideration and use existing infrastructure as an algorithm so that we can detect a tumour at an early stage and save a lot of money and consume less time. Adding to this, the mortality rate can be reduced by early detection of tumours. Humans are living in an age where health becomes a very crucial factor in order to maintain and then survive thus, having such a biomedical system can be highly efficient in terms of money and time.}\\

{The solution is an easy and simple brain tumour detection system that uses CNN - convolutional neural network algorithm and feeds on image input by the user which then segments the image and applies image processing and highlights if it detects brain tumour with proper highlighting.}\\

{The motivation behind selecting the particular project is the statistics observed till now. Brain Tumour incidence has increased more than 10\% over the past 20 years(paper 8). Such an increase in incidence puts higher pressure on the existing medical facilities. Thereby a simple solution is proposed that can help in solving the issue.}\\

{What our system does is it uses a end product ( web app ) which will have inputs as MRI scanned images of brain, it will perform digital image processing on the input image and will undergo CNN algorithm process. This involves validation and comparison with our trained model and based on its learning, it will classify whether it a tumour or not. If tumour, it will highlight the area affected inside the brain. Since brain tumour varies on the location it is affected, we can have a long run effect insight also from the same output. These insigts and reports can be stored on the same web application and cab be used further to compare the change in earlier records of the same patient also. }\\

{Overall, in a nutshell, early information on the brain tumour and how well it can be diagnosed can be obtained.}\\

The report is divided into 6 chapters. Chapter 1 deals with the basic introduction of the project. The chapter also deals with what a brain tumour actually is. Furthermore, problem motivation for the project and an overview of the project system is provided. Chapter 2 deals with the Literature Survey conducted. The literature survey highlights papers that give an insight into different algorithms, techniques and processing that can be used to detect brain tumours. Chapter 3 deals with Problem statements, Outcomes, Requirement Analysis, Impact Analysis and Limitations. A brief has also been given on Professional Ethics to be followed during the implementation of the project. Chapter 4 deals with the actual project implementation. Here a summary as to what resources are required actually, the basic block diagram have been discussed. Furthermore, the Detection Model and Webapp have been discussed in brief with flowcharts and algorithms. A model summary used has also been provided. At the end images of the proposed layout have also been attached for ease of understanding. Chapter 5 deals with the results obtained from the detection model, as well as results from the data augmentation. Chapter 6 deals with the future scope and conclusion.\\
%
