\begin{center}
{\huge \bf Abstract}
\end{center}
\vspace{20mm}
%\hspace{0.5 cm} 

Brain tumours, in medical terms are the intentional or unintentional growth of mass cells which hamper the conventional functioning of the shape of brain. For correct diagnosis and efficient treatment planning, it is necessary to detect the brain tumour in the early stages. The tumour within the brain is one of the most dangerous diseases and might be diagnosed easily and reliably with the assistance of detection of the tumour using automated techniques on MRI Images. Positron Emission Tomography, Cerebral Arteriogram, spinal tap, Molecular testing are used for tumour detection. Digital image processing plays an important role in the analysis of medical images. Segmentation of tumours involves the separation of abnormal brain tissues from normal tissues of the brain. Over few past years, various researchers have proposed semi and fully automatic methods for the detection and segmentation of Brain tumours. The motivation behind this project is to detect neoplasm and supply better treatment for the suffering. The objectives for the project are to develop an end-product (Web Application) that can be installed at hospitals. To facilitate this a detection model is developed that may accurately predict if an uploaded MRI scan of brain shows it is affected by tumour or not. To implement the project a Convolutional Neural Network(CNN) was used to define the model. Transfer Learning is implemented in order to efficiently train the model. The data-set used is split into 3 sets which are train, test and validation, in the ratio 80:10:10. The model is meant to be trained for 12 epochs. Callbacks also have been given to automate the model save process. The test accuracy of 97\% is achieved. This trained model will be connected with an online Application via API. Within the proposed Web App the user is having access to four routes which is welcome page and this contains information about the system, second route is information and awareness about the brain tumour in medical terms, third is detection page, where the trained model is deployed. The user is able to provide an input image, MRI images in our case, and last route is the team information. Images which are fed to the model route will be processed by the developed convolutional neural network which is able to then confirm if a tumour is present or not and intimidate the user for the same through an output Display. The advantage of using this system is that it will automate the detection process, and ease the workload of the hospital staff. However for the advantage to become a reality, careful selection of accurate data is needed, else there is a chance of false results.\\ 

{\bf{Keywords:}}\\ Brain Tumour Detection, Medical Image Processing, Brain tumour, MRI, CNN, Web Application, Case History 