\begin{center}
{\huge \bf Abstract}
\end{center}
\vspace{20mm}
%\hspace{0.5 cm} 

Brain tumours are a life-threatening problem and hamper the conventional functioning of the shape. For correct diagnosis and efficient treatment planning, it's necessary to detect the brain tumour in the early stages. The tumour within the brain is one of the most dangerous diseases and might be diagnosed easily and reliably with the assistance of detection of the tumour using automated techniques on MRI Images. Positron Emission Tomography, Cerebral Arteriogram, spinal tap, Molecular testing are used for tumour detection. Digital image processing plays an important role in the analysis of medical images. Segmentation of tumours involves the separation of abnormal brain tissues from normal tissues of the brain. Within the past, various researchers have proposed semi and fully automatic methods for the detection and segmentation of tumours . The motivation behind this project is to detect neoplasm and supply better treatment for the suffering. Our objectives for this project are to develop an end-product (Web Application) that can be installed at hospitals and within the end user's device. To facilitate this a detection model is developed that may accurately predict if an uploaded MRI scan contains a tumour or not. To implement the project a Convolutional Neural Network(CNN) was used to define the model. One Hot Encoding is employed to encode the pictures in order to better train the model. The dataset used is split into 3 sets which are train, test and validation, in the ratio 70:15:15. Furthermore, a sequential model is employed to define the model layers. The model is meant to be trained for 40 epochs. Callbacks also have been given to automate the model save process. The test accuracy of 92\% is achieved. This trained model will be connected with an online Application via API. Within the proposed Web App the user is first asked to log in, with the option to register just in case the user has accessed the web app for the first time. Once the user has logged in within the system a random function will provide different facts at different intervals of your time. Subsequently, the system redirects to the "Our Services" Page from where the user can do three things. The primary function is the in-built Detection Service. The users can go through the normal methods already existing. Existing users can also go through the access history. This will display any medical history of the patient. When the user uses the Detection Service, he's able to provide an input image, MRI images in our case. These images will be onto the present knowledge base ( Developed Model ) which is able to then confirm if a tumour is present or not and intimidate the user for the same through an output Display. The advantage to using this system is that it will automate the detection process, and ease the workload of the hospital staff. However for the advantage to become a reality, careful selection of accurate data is needed, else there is a chance of false results.\\ 

{\bf{Keywords:}}\\ Brain Tumour Detection, Medical Image Processing, Brain tumour, MRI, CNN, Web Application, Case History 