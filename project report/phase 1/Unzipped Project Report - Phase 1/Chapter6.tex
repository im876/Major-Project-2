\chapter{Conclusions and Future Scope}
\section{Conclusions }
Implementation of a Deep Learning Model which detects whether there is brain tumour present or not is observed. A total of 6 versions based on CNN are developed with various parameters into consideration. Based on these results and further testing the model with the best accuracy gives an system accuracy of 92\%. This result is verified using model evaluate function as well as the manual image set testing. Input image is passed onto the trained model which then determines whether image contains tumour or not. As per the inference by the model the output will be displayed on the web application accordingly.
\section{Future Scope }
\begin{itemize}
    \item The particular trained model can be further trained to detect tumour of initial stages.
    \item The particular model can be trained further to detect which type of brain tumour is present inside the body.
    \item The system in it’s base form can be trained further to detect the tumour even more efficiently
    \item The entire Web Application can be further developed to understand more types of human body scans such as X-ray, CT - Scans etc. So as to become a full fledged health assessment portal.
    \item Such web application could be installed at hospitals which will then connect patients and the hospital staff virtually.
    \item The system can be upgraded to detect more types of tumours such as ovarian, breast, skin, lung tumours etc.
\end{itemize}
