\chapter{Literature Survey}

1) A system that detects brain tumour using Convolutional Neural Network (CNN) is proposed in \cite{ref1}. The paper mentions the dataset used for training. The same data has been used for the development of the system proposed in this project. The authors mention various other steps implemented which include image augmentation, pre processing, segmentation of the skull, which can be implemented for better training of the model. The authors have used a custom architecture of CNN which provides an efficiency of 95.3\% on test images. The model used by the authors is as follows : 
\begin{figure}[H]
\includegraphics[scale=1]{Photos/paper1_model.PNG}
\caption{Paper 1 Model Summary} \label{fig:ishan}
\end{figure}

2) Insights into various symptoms, causes and statistics of brain tumour is identified from \cite{ref2} , \cite{ref3} and \cite{ref5}. This data will be used into the information module of the web application.\\ 

3) A survey on major deep learning systems that have been implemented to do various types of brain tumour analysis can be inferred from \cite{ref4}. This survey gives an insight into the existing detection models. The paper further gives an insight into datasets, techniques into consideration as well as the evaluation of each particular paper considered into the survey.\\

4) A tumour detection system that used image segmentation as the base technique is described in breif in \cite{ref5}. The authors use a clustering based approach using Self Organizing Map (SOM) algorithm. The detection works in two phases, the first being image acquisition and pre processing. The second phase is image segmentation which basically segments principle tissue structures in the input images. The authors propose a new MR image segmentation method based on fuzzy C-Means clustering algorithm for the Segmentation which is unsupervised. \\

5) Comparision between nonlinearity reduction and linear method is done in \cite{ref6}. The authors experiment to check if the non linear method provide better results (unsupervised classification) of MRS brain tumour data. Data reduction is performed using Laplacian eigenmaps (LE) or independent component analysis(ICA). This was then followed by k-means clustering or agglomerative hierarchical clustering (AHC) for unsupervised learning to assess tumour grade and for tissue type segmentation of MRSI data. Based on the results obtained LE method is promising for unsupervised clustering to separate brain and tumour tissue with automated color-coding for visualization of 1 H MRSI data after cluster analysis.\\

6) A survey on different segmentation techniques is given \cite{ref7} conduct a survey on different segmentation techniques. The paper provides different papers, their datasets, and various parameter information that gives a brief insight into different algorithms that can be implemented to detect brain tumour. The survey is divided into the following categories: Thresholding techniques, Region growing techniques, Edge based techniques, Clustering techniques, Watershed technique, and Deformable model-based techniques.\\

7) A image segmentation based approach for tumour detection is explained in brief in \cite{ref8}. From the paper it can be inferred that Magnetic Resonance Imaging (MRI) scans are the best medical scan for the training and detection of brain tumour. A deep insight is given into the different parameters associated with image enhancement using different filters and techniques. Furthermore insights are also provided into segmentation techniques.

8) A custom model named as Deep Wavelet AutoEncoder Model (DWAE Model) is proposed in \cite{ref9}. A high pass filter is used to show the heterogeneity of MRI images. After preprocessing is performed,  the DWAE Model analyzes the pixel pattern of the input scan, and classifies the tumour accordingly. The test results show that the model achieves an accuracy of 99.3\% with a validation loss of 0.1\%.